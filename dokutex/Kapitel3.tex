
\chapter{Resümee}
\label{Resümee}

\section{Diskussion}
Die Darstellung von Parameterkurven ist in Unity gut möglich. Lediglich die Skalierung muss angepasst werden. Es gibt verschieden gut geeignete Parameterkurven. In diesem Projekt wurden ein Kreis, eine archimedische Spirale und eine Parabel gewählt.
Der Kreis ist gut darstellbar, ebenso die Spirale. Eine Pfadanimation lässt sich relativ einfach implementieren. 

Die Darstellung einer Parabel ist \glqq instabil \grqq{}. Das Intervall sollte geschickt ausgewählt werden. 
Im Allgemeinen lässt sich mit genug Zeit und Expertise ein Teil der mathematischen Konzepte modellieren bzw. darstellen. Der Spieler schlüpft in eine Art übergeordnete Rolle und kann die erzeugten Objekte frei, selbständig und wahlfrei betrachten.

Es stellte sich heraus, dass bei einer Pfadanimation am Kreis oder der Spirale die Darstellung und die Bewegung des Spielers intuitiv zweckmäßig implementiert wurden. 
Die Parabel hingegen stellt vor Probleme. Der Spieler würde sich ausschließlich nach oben bewegen.

Aus diesem Grund ist in der Szene der Parabel keine automatische Pfadanimation implementiert. Im Gegenteil, der Spieler ist in der Lage sich mittels Teleportation an eine beliebige Stelle auf der Plane der Szene zu bewegen. So kann die Form von einem beliebigen Punkt betrachtet werden.

Verscheidene Entscheidungen erwisen sich als unzweckmäßig:
\begin{enumerate}
	\item Graph in XZ-Ebene darstellen. \\
			Bei der Darstellung der Parabel in der XZ-Ebene von Unity (\glqq etwa seitlich auf dem Boden liegend\grqq{}) wurde deutlich, dass die Parabel zu schnell von der Ebene rutschte. Ihre Dimensionen wachsen sehr schnell. 
		
			Daraufhin wurde die Parabel senkrecht auf die Plane gestellt.
	\item Eigenimplementierung eines Tracked\_Object scripts. 
	
	Die eigene Implementierung eines Scriptes zum verfolgen von Objekten endete in einer Sackgasse.
	Es wurde ein Script aus dem SteamVR Plugin verwendet.
\end{enumerate} 



\section{Rückblick}

In der Nachbetrachtung wurde festgestellt, dass für dieses Thema zumindest Grundwissen über die Entwicklungsumgebung und Kentnisse in der Sprache C\# vorausgesetzt werden sollten. Ist dies nicht der Fall, so ergeben sich große Schwierigkeiten bei der Umsetzung.

Der Workload, vor allem im Bereich \glqq Erlernen einer Programmiersprache in einem speziellen Kontext\grqq{}, wurde stark überschätzt.   

Es wäre eleganter, wenn die Pfadgeneration mittels eines LineRenderers durchzuführen und darauf Meshes zu erzeugen. Dies würde helfen eine optisch ansprechendere Szene zu erstellen.

Bei einer zukünftigen Anwendung würde ich die Szenen aus der VR-Sicht planen und früher mit der Vive testen.

Bei der Wahl der IDE hätte ein früheres wechseln auf Rider die Entwicklung beschleunigt.

Die Darstellung des Graphen in der entsprechenden Szene ist nicht gut gelungen, da die Dimensionen sehr schnell wachsen und eine entsprechde optische Metapher fehlt bzw. die Darstellung wenig detailliert ist. 





