
\chapter{Szenen}
\label{Szenen}

Im vorliegenen Projekt wurde für jede Parameterkurve eine eigene Szene erstellt, damit die einzelnen Kurven gegeneinander abgegrenzt sind.


\section{Circle}
\label{Kreis}

Die Szene enthält den einen Kreis, der beim Start erzeugt wird. Dieser wird mit konstanter Geschindigkeit umlaufen. Der Kreis ist mit kleinen Pfosten markiert, die eine Gasse bilden. 

\subsection{Elemente}

\textbf{Plane: } Die Grundfläche der Szene ist eine 4,5 x 3,5 Meter großes 3D-Objekt vom Typ \emph{Plane}.

\textbf{Capsule: } Der Spieler wird mit Hilfe eines GameObjects in Form einer Capsule repräsentiert. Das Camera Rig ist mit Hilfe eines Scripts an dieses Objekt gebunden. Die Capsule hat die Markierung: Player.

\textbf{Brick: } Bricks sind Prefabs, die zur Laufzeit erzeugt werden. Ein einzelner Brick ist konfiguriert, ein Script instanziert Kopien davon.

\textbf{PlayerC: } PlayerC Objekte liegen ebenfalls als Prefab vor. Sie werden in Abhängigkeit der Bricks erzeugt und beschreiben die Positionen, die während der Pfadanimation verwendet werden. PlayerC sind auch GameObjects, haben aber keinen Mesh-Renderer also sind sie unsichtbar. Sie sind als "waypoint" getagged.

\textbf{Canvas: } Auf der Canvas befinden sich zwei Buttons, die dem Szenenwechsel dienen. Die Canvas befindet sich in der Mitte des Kreises. Zur Auswahl einer anderen Szene mit dem Laserpointer des linken Controllers auf den Button zeigen und das Touchpad des Controllers klicken. 


\subsection{Scripte}
\label{Circle_Scripts}

\textbf{Instantiate.cs}
Dieses Script dient dazu, den Kreis und die PlayerC Punkte zu erzeugen. In der Awake()-Methode werden mittels einer For-Schleife
die Bricks und die PLayerC Objekte erzeugt. Die Plane wird anhand ihres Namens \glqq Boden{}\grqq{} gesucht. Relativ zum Boden wird die Y - Höhe der Bricks berechnet.
Zusätzlich werden die Positionen der einzelnen PlayerC in ein Array geschrieben, dies dient der späteren Pfadanimation.
Es befinden sich noch zwei weitere Funktionen im Script. Diese werden bei jedem Brick bzw. PlayerC aufgerufen. Die X.Koordinate einer Paramterkurve wird über die Funktion $ f(t) = \cos(t) $ definiert. Für Z gilt $g(t) = \sin(t) $ analog. In diesem Koordinatensystem beschreibt Y die Hochachse.

\section{Quantoren und Normalform}
\
\section{Resolution}
\label{Resolution}

\section{Automatische Theorembeweiser}
\label{Automatische Theorembeweiser}


\section{Anwendungen}
\label{Anwendung}
