% !TeX root = Bachelorarbeit.tex
\chapter{Einleitung}
\label{Einleitung}

\section{Motivation}
\emph{Bustracker} ist eine Fallstudie im Projekt \glqq Digitale Kommune\grqq{} der Entwicklungsagentur Rheinland-Pfalz e.V. \emph{Bustracker} soll die Möglichkeit demonstrieren in ländlichen Gebieten den Schulweg der eigenen Kinder zu beobachten. 

In vielen Gemeinden müssen Schüler oder sogar Kindergartenkinder über weite Strecken mit dem Bus zu ihrer Betreuung oder Lerninstitution fahren. In den seltensten Fällen sind noch Betreuungspersonen im Bus vorhanden. Die gängige Praxis, die Kinder durch Eltern mit dem \gls{gls:kfz} an die Schule/den Kindergarten zu bringen, führt zunehmend zu einer Verkehrsüberlastung in den Bereichen um die Betreuungs- bzw. Lernstätte. Die Eltern versuchen, teilweise unter Ignoranz der Straßenverkehrsordnung, den Weg zum Bus oder zur Schule zu verkürzen. 
Damit geht eine erhöhte Gefahr für die Kinder einher, was die Eltern eigentlich durch den Transport per \gls{gls:kfz} zu vermeiden versuchen. 

\emph{Bustracker} soll Eltern die Möglichkeit geben ihre Kinder guten Gewissens mit dem Bus fahren zu lassen.
Plötzlich auftretende Ereignisse, wie z. B. eine Panne am Bus oder eine Straßensperrung können früher erkannt werden, da die Eltern mit \emph{Bustracker} den Schulweg ihres Kindes verfolgen können. Dadurch kann früher reagiert und eine andere Transportmöglichkeit organisiert werden. Die Eltern wissen jederzeit wo sich ihr Kind befindet.  

\emph{Bustracker} implementiert verschiedene Technologien und dient zum Testen und Entwickeln des Konzeptes.


\section{Überblick}
Im Kapitel \textbf{\nameref{Verwendete Technologien}} wird auf die in diesem Projekt verwendeten Softwaretechnologien eingegangen. Die verwendeten Werkzeuge sind ebenfalls in diesem Kapitel beschrieben.

Das Kapitel \textbf{\nameref{Organisation}} beschreibt die Art und Weise wie die Entwicklung organisiert wurde. Der komplette Verlauf ist dort ebenfalls beschrieben.

\textbf{\nameref{Frontend}} erklärt den Aufbau und die Anwendung der in dieser Arbeit entwickelten App.

\textbf{\nameref{Backend}} enthält die Erklärungen und Beschreibungen zu den einzelnen Teilen der Software.

Im Kapitel \textbf{\nameref{Fazit und R\'{e}sum\'{e}}} wird das Projekt kurz zusammengefasst. Vor- und Nachteile sowie gesammelte Erfahrungswerte werden im Kapitel \textbf{\nameref{Lessons learned}} besprochen. In diesem Kapitel werden ebenfalls Lösungsstrategien für die während der Entwicklung aufgetretenen Probleme dargestellt. Im Anschluss daran wird im \textbf{\nameref{sec:Ausblick}} weitere Verbesserungsvorschläge oder weitere Modifikationen der App besprochen.
